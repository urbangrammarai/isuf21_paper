\section{Conclusions}
\label{sec:concl}

% summary and punchline
This paper identifies the scalability issue of traditional urban morphology and presents
a method of characterisation of space based on its form component, one which embraces
urban morphometrics and offers scalability to large extents. That is illustrated on the
case study of Great Britain, which environment has been classified into 19 types of
form-based spatial signatures. Those can be organised into three macro groups of
countryside, suburban low density development and dense city centres. The spatial
composition of city centres then tells a story of hierarchy of British cities with
London being the only one containing the most \textit{urban} of signatures.

Compared to other scalable methods, spatial signatures build on both street networks and
building footprints using the inclusive set of morphometric characters aiming to capture
a wide spectrum of variables, minimising a selection bias incurred by small sets. The
method paves the way towards large scale urban morphology which is able to discuss
urban form across nations while still retaining its most natural neighbourhood scale.
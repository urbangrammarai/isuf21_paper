\section{Introduction}
\label{sec:intro}

Identification of recurring patterns in the built environment is deeply embedded within
all schools of urban morphology. We can discuss Conzen's Plan Unit \citep{conzen1960},
which evolved into Whitehand's morphological region within the British
historical-geographical tradition, aiming to capture internally homogenous areas of a
shared origin and spatial character \citep{oliveira2020}. At the same time, we may talk
about \textit{tessuta urbana} or \textit{urban tissue}, stemming from the Italian school
of typo-morphology \citep{caniggia2001}, and sharing the notion of internal homogeneity,
just detected by different methods reflecting the architectural perception of space more
than geographic. However, it is a complicated pursuit to scale up morphological studies
of these kinds without losing too much information. Traditional schools (both
historical-geographical and typo-morphological) are generally not able to scale their
methods to larger areas while keeping the detail. Another well-established school of
urban morphology - Space Syntax, based on the work of Hillier and Hanson
\citep{hillier1996}, is different and can be considered scalable to metropolitan or even
national extents \citep{spacesyntaxlimited2018}. However, Space Syntax at these scales
is limited to an analysis of street networks and their configuration, completely
omitting patterns formed by plots, buildings, and open spaces. The same can be said
about a broader range of network-based methods like Multiple Centrality Assessment
\citep{porta2010} - while they can be scaled, their insight is inherently limited by the
limited data input.

The recent growth of purely quantitative methods of urban form analysis, often nicknamed
\textit{urban morphometrics}, and their ability to scale without losing too much detail
is opening a range of opportunities to give urban morphology a toolkit to analyse
recurring patterns at metropolitan and national extents. After the first explorations in
the works of \cite{gil2012} or \cite{hamaina2012a}, the methods are starting to mature
as illustrated by recent publications of Multiple Fabric Assessment by
\cite{araldi2019}, a series of element-based typologies by \cite{berghauserpont2019a},
gridded classification by \cite{jochem2020} or hierarchical model following the
biological methods of taxonomy creation by \cite{fleischmann2021a}. All share a similar
approach, based on the an initial set of measured characters capturing the individual
aspects of form-based patterns and subsequent unsupervised classification. As all
methodological steps can be expressed as computer algorithms, they can potentially scale
to nation-wide analyses, as already shown by Jochem et al. in the case of Great Britain
\citep{jochem2021tools}. Is it to be noted that each of the existing methods has its
limitations, often linked either to the a spatial unit that does not ensure internally
homogenous urban patterns (Jochem er al., Araldi and Fusco), dependency on rarely
available data (Berghauser Pont et al.) or a limited number of measured characters,
which may omit some aspects of patterns and introduce selection bias in the method
(Berghauser Pont et al.).

This paper presents \textit{spatial signatures}, a method of characterisation of space
based on its form and function, and focus specifically on its form component, able to
delineate internally homogenous areas of urban form based on an extensive set of
morphometric characters. The method is applied in the case of Great Britain, deriving an
exhaustive classification of both built and non-built environment. The remainder of the
paper outlines the method (Section \ref{sec:meth}), including the introduction of the
Enclosed tessellation as the spatial unit, presents resulting classification (Section
\ref{sec:res}), and discussed its implications on the analysis of urban form (Section
\ref{sec:disc}).
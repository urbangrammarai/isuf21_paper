\section{Discussion}
\label{sec:disc}

% - signatures in the context of morphological studies
% -- conceptual similarity with morphological region, urban tissue and similar
% -- differences between the concepts and resulting complimentarity
    % different scales, different purposes, different questions
Spatial signatures, either based on a combination form and function or one components
only, aim to identify and delineate recurring patterns of the built environment. As
such, they are deeply embedded in the tradition of urban morphology directly related to
established concepts like morphological region, urban tissue and similar. All aim to
determine which areas of cities are internally more similar that they would be to any
other place. At the same time, the concepts and their methods are not same and will
undoubtedly produce a variation in the classification of urban form. Where
historico-geographical morphological region would detect original development and its later
expansion following the same pattern as two distinct areas (due to a different
historical origin), form-based signatures would result in a single area as the pattern
is homogenous. Which of the two is the \textit{correct} answer then fundamentally
depends on the question asked. Each of the concepts and related methods of their
analysis are bound to different scales, purposes and questions and are complimentary
rather than competing.


% - reproducibility of morphological studies
% -- morphometric methods can be algorithmised, hence easily replicable and reproducible
Unlike traditional urban morphology that based mostly on qualitative methods requiring a
high degree of expert knowledge and interpretation, morphometric studies offer a high
degree of reproducibility and replicability. The methods are often fully written as
Python (as in case of this research) or R code, which given the same data on input
produces the same results and given similar data for a different case study area should
produce comparable results. The stability and reliability of the method is especially
important in the policy-making context which needs to be consistent across years and
updates of supporting morphological data.

% - limits
% -- in the context of morphometrics
% --- suboptimal data input (individual buildings)
%   --- national extent and hence resolution of clustering - local differences may be
%   smoothed out as insignificant from the national perspective
% --- precision of boundary placement

Morphometric approaches have their limitations, related to the the dependency on the
quality of input data as well as the behaviour of used algorithms. Suboptimal data
input can significantly affect the performance of signature detection. Take, for example,
the quality of building footprints as one of the most utilised data source. While there
is a wide range of reliable, mostly governmental or municipal, sources of data, that
does not apply to every place and dataset. Some, like OpenStreetMap, provide highly
inconsistent data with some buildings drawn in a high detail or even subdivided into
multiple parts based on variable building height, other do not distinguish between
individual buildings and provide a single geometry for the whole block composed on
multiple street fronts and buildings.

Algorithms used to classify features into signatures also make a difference, especially
on the national scale shown in this paper. The classification and to a degree a
\textit{resolution} of clustering is done on a national level, that may result in local
differences being smoothed out as insignificant from the national perspective. The zoom
into individual neighbourhoods then may seem to be coarse in terms of identified
signatures. Same applies to the placement of the boundaries between signature types.
Trained urban morphologist or urban designer would likely question why one building
belongs to a type A while the other within the same block to a type B but we have to
keep in mind the algorithmic nature of the work providing a prediction more than
anything else.

% - scalability of morphological studies
% -- we can ask questions about larger patterns that we did before
% -- we can look into cross-national similarities and tendencies
However, the significant difference between the presented method and traditional
morphology is the scalability of the analysis. With national-scale data we can ask
questions about much larger patterns than we did before. Obtaining the analysis covering
the same extent as the one just presented in the same level of detail using
historico-geographical or typo-morphological methods would be simply an unfeasible task.
With morphometric methods we can start looking into cross-national similarities and
tendencies in urban development on the neighbourhood-like scale signatures provide.
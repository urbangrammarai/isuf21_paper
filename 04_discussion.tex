\section{Discussion}
\label{sec:disc}

% - signatures in the context of morphological studies
% -- conceptual similarity with morphological region, urban tissue and similar
% -- differences between the concepts and resulting complimentarity
    % different scales, different purposes, different questions
The spatial signatures framework, either based on a combination form and function or a
single component, aims to identify and delineate recurring patterns of urban
spaces. As
such, it is deeply embedded in the tradition of urban morphology, directly related to
established concepts like morphological region, urban tissue and similar. All
of them aim to
determine which areas of cities are internally consistent. The differences
between these perspectives stem from the methods they rely on, which
undoubtedly results in variation in the classifications of urban form they
produce. Where
historico-geographical morphological region would detect the original development and its later
expansion following a similar pattern as two distinct areas (due to a different
historical origin), the morphosignatures would class them in a single area as the pattern
is homogenous. Which of the two is the \textit{correct} answer then fundamentally
depends on the question asked. Each of the concepts and related methods of their
analysis are bound to different scales, purposes and questions, and are complimentary
rather than competing.


% - reproducibility of morphological studies
% -- morphometric methods can be algorithmised, hence easily replicable and reproducible
Unlike traditional urban morphology that is based mostly on qualitative methods requiring 
 expert knowledge and interpretation, morphometric studies offer a high
degree of reproducibility and scalability. Methods are often embedded in
computer code release under open licenses which, given the same input data,
produces the same results. This reliability of the method is especially
important in the policy-making context which needs to be consistent across years and
updates of supporting morphological data.

% - limits
% -- in the context of morphometrics
% --- suboptimal data input (individual buildings)
%   --- national extent and hence resolution of clustering - local differences may be
%   smoothed out as insignificant from the national perspective
% --- precision of boundary placement

Morphometric approaches have their limitations, related to the the dependency on the
quality of input data as well as the behaviour of used algorithms. Suboptimal
input data can significantly affect the performance of morphosignature detection. Take, for example,
the quality of building footprints as one of the most utilised data source. While there
is a wide range of reliable, mostly governmental or municipal, sources of data
for some regions of the world, this is not the case everywhere. Some sources, like OpenStreetMap, provide highly
inconsistent data, with some buildings drawn in high detail and others in very
coarse ways. Fortunately, technology is moving very rapidly and advances in
data collection and creation (e.g. building footprints derived from satellite
imagery) may change this landscape for the better in the coming years.

The algorithms used to classify features into morphosignatures also make a
difference, as does the scale at which an analysis is run.
Our classification and, to a degree, the
\textit{resolution} at which the clustering is done is national. This choice may result in local
differences being smoothed out if they are less relevant from the national
perspective. Zooming into individual neighbourhoods, one may find some of the
decisions made by the algorithm too coarse. The same applies to the placement of the boundaries between morphosignatures.
The trained urban morphologist or urban designer might question why one building
belongs to a type A, while the other one within the same block is assigned to
type B. It is important to
keep in mind the scale and goal of the classification, which is then embedded
in the algorithmic choices.

% - scalability of morphological studies
% -- we can ask questions about larger patterns that we did before
% -- we can look into cross-national similarities and tendencies
Our main contribution with respect to traditional
morphological studies resides in the scalability of the analysis. With national-scale data we can ask
fundamentally new questions about the emergin patterns than is possible with
smaller scales. Obtaining the analysis covering
the same extent as the one just presented at the same level of detail using
historico-geographical or typo-morphological methods would be simply an unfeasible task.
Morphometrics allows us to begin considering cross-national comparisons that
consider detail at neighbourhood-like scale.
%
As we develop more fine grained and scalable classifications, an
important challenge will be to devise ways that allow us to blend the
insights we can gain at this scale with the body of knowledge developed over
time from more qualitative approaches.

\section{Data and code availability}
The repository, containing reproducible code in Jupyter notebooks is available at
\url{https://urbangrammarai.github.io}. Guidelines on how to access and
download the resulting classification are available in the same website.

\section{Results}
\label{sec:res}

% - 19 classes within 3 groups
The resulting classification of Great Britain identified 19 types of form-based
signatures that can be organised into three macro groups: countryside, suburban low
density development and dense city centres. It is a result of two hierarchical steps
based on K-Means clustering, where the first resulted in 7 classes, of which the two
most urban have been further clustered into 6 and 8 lower-level classes (after a removal
of outlier classes).

% -- countryside
% --- brief portrait
Countryside macro group is composed of four signature types covering large-scale open
spaces from agricultural land in southern England to vast natural areas of Scottish
Highlands. The urban development in this group is limited to small villages or hamlets.
All four classes are a result of the first clustering step.

% -- suburbs
% --- brief portrait
Second macro group covers suburban low-density development areas, taking up most of the
area of british cities. We can identify 9 types of signatures, originating in two
classes from the first step. The range of types stretches from sparse single-family
housing on the peripheries of cities, planned residential developments of 20th century
to predominantly industrial areas. The types differ in many aspects, from the overall
built-up density and related geometry of both enclosed tessellation and enclosures
(both affecting the description via many morphometric characters) to connectivity of
street networks or their solar orientation.

% -- city centres
% --- brief portrait
Final macro group comprises of dense town and city centres, all originating from the
single top level cluster. These signatures reflect the main hubs of activities in each
larger settlement. In some cases, they are all located in the same central areas, while
in others some local district nodes show up (e.g. Liverpool). All six types can be
arranged according to their level of \textit{urbanity} and tend to form concentric
rings.

% - Concentric character of British centers and centre hierarchy
% -- from London to Oxford
Two types are exclusive to
London's city centre (roughly around Soho) and are not present anywhere else in the
whole country. The remaining are present in other places and their presence can encode
the \textit{urbanity} of each city or town. For example, Birmingham as the second
largest city in the country contains four types of central signatures, compared to six
in London. Scottish capital Edinburgh contains only three (like many other cities across
the country (e.g. Manchester or Glasgow)). Smaller cities like Southampton have only two
types and towns as Oxford are limited to a single central type. The presence of types is
not accidental -- smaller cities lack the most urban types, while all of them contain the
least urban type.

% -- Milton Keynes does not have a morphological centre
Notable is the case of Milton Keynes, a new town built since 1960s on the green field
with a target population of 250 000 (currently at 230 000). Its development followed a
carefully designed masterplan, laying out the whole city. However, the resulting
structure is very different from any other city in the country as none of the signatures
encoding urban centre is not present in Milton Keynes. We could say, that it does not
have a morphologic centre.

% ADD A BUNCH OF FIGURES TO THE SECTION